
% das Papierformat zuerst
\documentclass[a4paper, 11pt]{article}
\usepackage[ngerman]{babel}
\usepackage[utf8]{inputenc}
\begin{document}

\tableofcontents
\newpage

% hier beginnt das Dokument


\section{Beschreibung des Umfeldes}

Dies ist ein Satz.

\section{Darstellung der Aufgabe}

Konkret sollte im Rahmen des Infoboard Projektes eine Anwendung entwickelt werden, mit welcher sich von Redakteuren erstellte Inhalte unterschiedlicher Formate auf den in den Arbeitsräumen installierten Bildschirmen anzeigen lassen. Neben der eigentlichen Ausgabe wird dafür eine weitere Oberfläche benötigt, über die Inhalte einpflegt werden können. Diese sollte durch Zugangsdaten geschützt sein.

- weitere Module zuschalten

- 

\section{Beschreibung von Lösungsmöglichkeiten}

Dies ist ein Satz.

\section{Das Lösungskonzept}

lol.

\section{Ausgewählte Realisierungsansätze}

Spring

Docker

Jenkins

\section{Zusammenfassung, Ausblick, Reflektion}

% das ist wohl jetzt das Ende des Dokumentes
\end{document}